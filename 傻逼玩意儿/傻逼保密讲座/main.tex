\documentclass[12pt]{article}
\usepackage[utf8]{inputenc}
\usepackage{ctex}
\usepackage{indentfirst}
\usepackage{geometry}
\RequirePackage{titlesec}
\RequirePackage{fancyhdr}
\RequirePackage{xeCJK}
\RequirePackage{CJKnumb}
\setCJKmainfont[
  Path = ../../fonts/ ,
  Extension = .ttf ,
  BoldFont = SimHei ,
  ItalicFont = simkai ,
]{SimSun}

\titleformat*{\section}{\large\bfseries}
\titleformat*{\subsection}{\normalsize\bfseries}
\titleformat*{\subsubsection}{\normalsize}

\titlespacing{\subsection}{1em}{*4}{*1.5}
\titlespacing{\subsubsection}{1em}{*4}{*1.5}
\geometry{top=1in,bottom=1in,left=1in,right=1in}

\renewcommand\thesection{\CJKnumber{\arabic{section}}、}
\renewcommand\thesubsection{\arabic{subsection}.}
\renewcommand\thesubsubsection{(\arabic{subsubsection})}

\title{\huge\textbf{保守秘密,慎之又慎} \\ \Large{观保密教育有感}}
\author{计算机学院 \hspace{1em} 保密管理 \hspace{1em} 武自厚 \hspace{1em} 20336014}
\date{\today}

\begin{document}

\maketitle

众所周知,中山大学计算机学院挂着三块牌子,第一块是“计算机学院”,第二块是“软件学院”,
而第三块,则是最为特殊的“国家保密学院”。我有幸在这样一个特殊的学院就读保密管理这样
特殊的专业,在管理与技术并举的培养方案中就读。

保密工作是一项管理工作。它是指从国家的安全和利益出发,为达到保护国家秘密的目的,
通过一系列手段和措施约束、规范人和组织的行为,使国家秘密能够在一定的时间内控制在一定的知悉范围内,
防止被非法泄密和使用。涉密人员只有把保密管理当作保饭碗、保前途、保家庭幸福的事情来做,
才能保障改革开放和社会主义事业顺利进行。

保密工作是一项重要工作。它是从国家的安全和利益出发,为达到保护国家秘密的目的,通过一系列手段和措施约束、规范和组织人们的行为,使国家秘密能够在一定的时间内控制在一定的知悉范围内,防止被人蓄意利用和非法泄密。

保密工作担负着“保安全,保发展”的重任,历来受到党和国家的高度重视。在革命战争年代,保密就是保生命,而在和平建设年代,保密就是保发展。随着我国在政治、军事、经济、科技等领域日益增强,以及数字化、信息化、网络化的快速发展,保密工作面临着严峻形势和艰巨的任务。做好新形势下保密工作,必须按照党的十八大以来党中央提出的新要求,突出重点,强化措施,牢固树立新的理念。

当今世界,伴随着综合国力的竞争,保密与窃密的斗争变得尖锐复杂。尤其是随着科学技术的发展,
利用高科技手段进行的窃密无孔不入。表现在手段上,更加高科技化和更具隐蔽性,
有通过网络植入木马病毒窃取电脑信息的,也有通过办公设备安装窃密工具的;
表现在行为上,分有意识泄密和无意识泄密,无意识泄密主要由于保密意识不强,
在日常工作中可能通过在涉密电脑与非涉密电脑之间交叉使用移动存储介质,
或将涉密计算机及网络接入互联网及其他公共信息网络等行为导致泄密。

一言以蔽之,就是毛主席说过的“保守秘密、慎之又慎”八个大字,纵使社会经济技术环境不断变化,
始终不变“党管保密”的铁律,是保密工作者孜孜不倦前仆后继的奋斗。
\end{document}
