\section{平面图}

\subsection{平面图的概念与性质}

通过以下例子可以导入平面图问题:

\begin{example}[电路板设计问题]
    电路板中的电路元件的连接导线互不交叉, 可以转化为
    ``要求图中的边不能相互交叉''.
\end{example}

\begin{definition}[平面图]
    如果能将图 \(G\) 画在平面上, 且所有边相互都不相交,
    则称 \(G\) 为\textbf{可平面图}. 这样的一种画法
    称为 \(G\) 的一种\textbf{平面嵌入}, \(G\) 的
    平面嵌入表示的图称为\textbf{平面图}.
\end{definition}

\begin{definition}[平面图的面]
    一个平面图\(G\)将平面分成若干个连通片, 这些连通片
    称为\textbf{区域}或\textbf{面}. 
    G的面组成的集合用\(\varPhi(G)\)表示.

    其中面积有限的面称为\textbf{内部面}, 其余称为
    \textbf{外部面}.

    在\(G\)中, 顶点和边都与某个给定面关联的子图,
    称为该面的\textbf{边界}. 某面\(f\)的边界中
    含有的边数(割边计算2次)称为面\(f\)的\textbf{次数},
    记为\(\deg f\)
\end{definition}

\begin{theorem}[仿握手定理]
    设\(G = (n,m)\)是平面图, 则:
    \[\sum_{f \in \varPhi(G)} \deg f = 2m\]
\end{theorem}

\begin{theorem}[平面图的Euler公式]
    设\(G = (n,m)\)是连通平面图, \(\phi\) 是
    \(G\)的面数, 则有:
    \[n - m + \phi = 2\]
\end{theorem}

\begin{deduction}
    设\(G\)是具有\(k\)个连通分支的平面图, 则有:
    \[n - m + \phi = k + 1\]
\end{deduction}

\begin{deduction}
    设\(G\)是具有\(n\)个点\(m\)个边\(\phi\)个面
    的连通平面图, 如果
    \(\forall f \in \varPhi(G):3 \le l \le \deg f\),
    则有:
    \[m \le \frac{l}{l-2}(n-2)\]

    反过来, 如果\(G=(n,m)\)是连通图, 且
    \[m > \frac{l}{l-2}(n-2)\]
    则\(G\)是不可平面图.
\end{deduction}

\begin{theorem}[两种不可平面图]
    \(K_{3,3}\)和\(K_5\)都是不可平面图.
\end{theorem}

\begin{deduction}
    设连通平面图\(G=(n,m)\), 若\(G\)的每个圈都是
    由长度是\(l\)的圈围成, 则:
    \[m(l-2)=l(n-2)\]
\end{deduction}

\begin{deduction}
    设简单平面图\(G\), 则:
    \[\delta(G) \le 5\]
    也就是说简单平面图不能以\(K_5\)为子图.
\end{deduction}

\begin{theorem}
    一个连通平面图是2-连通的, 当且仅当它每个面的边界都是圈.
\end{theorem}

\subsection{图的嵌入性问题简介}

\begin{theorem}
    如果图\(G\)是可平面的, 那么它也可以球面嵌入.
\end{theorem}

\begin{theorem}
    \(K_5\)和\(K_{3,3}\)都是可环面嵌入的.
\end{theorem}

\begin{theorem}
    所有的图都可以嵌入\(\R^3\).
\end{theorem}

\subsection{凸多面体与平面图}

\begin{definition}[凸多面体]
    如果一个多面体上任取两点, 之间的连线都在该多面体内,
    则称为凸多面体.
\end{definition}

\subsection{可平面性充要条件}

\begin{definition}[2度顶点收缩/扩充]
    将一个2度顶点消除且其关联的边合成一条, 称为2度顶点收缩;
    其逆过程称为2度顶点扩张.
\end{definition}

\begin{definition}[同胚]
    两个图\(G_1,G_2\)是同胚的, 如果 \(G_1 \cong G_2\),
    或者通过有限次通过2度顶点收缩/扩充能够变为同构的图.
\end{definition}

\begin{theorem}[Kuratowski]
    图\(G\)是可平面的, 当且仅当它不含与\(K_5\)或\(K_{3,3}\)
    同胚的子图.
\end{theorem}

\begin{definition}[基础简单图]
    给定图\(G\),去掉\(G\)中的环, 用单边代替重边, 得到的图
    称为\(G\)的基础简单图.
\end{definition}

\begin{theorem}
    图\(G\)可平面, 当且仅当它的基础简单图是可平面的,
    也当且仅当\(G\)的每个块可平面.
\end{theorem}

\begin{definition}
    设\(uv\)是简单图\(G\)的一条边. 去掉该边, 重合其端点, 删去
    由此产生的环和重边. 这一过程称为图\(G\)的初等收缩. 如果图\(G\)
    可以初等收缩为图\(H\), 则称\(G\)可收缩到\(H\).
\end{definition}

\begin{theorem}[Wagner]
    图\(G\)是可平面的, 当且仅当其不可收缩为\(K_5\)或\(K_{3,3}\).
\end{theorem}

\begin{theorem}
    至少有9个顶点的简单平面图的补图是不可平面的.

    \textit{注: 这个定理的条件是紧的}.
\end{theorem}
