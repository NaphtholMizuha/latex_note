\section{连分数}

\begin{definition}[有限连分数]
    记形如
    \[x_0 + \cfrac{1}{x_1 + \cfrac{1}{x_2 + \ddots}}, \quad \text{最深处为} x_n\]
    的数为 \(n\) 阶有限连分数.简单记为
    \[[x_0;x_1,x_2,\cdots,x_n]\]
    如果 \(x_i\) 都为正整数, 则称为简单连分数.

    \([x_0;x_1,x_2,\cdots,x_k] = \frac{P_k}{Q_k}\) 称为 第\(k\) 渐进连分数.

\end{definition}

\begin{definition}[无限连分数]
    假设在上一个定义中的数列\(\{x\}\) 是无穷序列, 则称其为无限连分数, 记为:
    \[[x_0;x_1,x_2,\cdots,x_l,\cdots]\]
\end{definition}

\begin{example}
    \(\sqrt{2}\) 可以写为无穷连分数:
    \begin{align*}
        \sqrt{2} &= 1 + \frac{1}{2 + \sqrt{2}} \\
        &= 1 + \cfrac{1}{2 + \cfrac{1}{2 + \sqrt{2}}} \\
        & \cdots \\
        &= 1 + \cfrac{1}{2 + \cfrac{1}{2 + \cfrac{1}{2 + \ddots}}} \\
        &= [1;2,2,2,\cdots]
    \end{align*}
\end{example}

\textbf{实数的简单连分数构造法} \hspace{1em} 给定实数 \(x\).
\begin{itemize}
    \item 令 \(x_0 := \floor{x}, r := x - x_0\)
    \item 如果 \(r = 0\), 则返回 \(x = [x_0]\). 否则令 \(x_1 := \floor{\frac{1}{r}}, r := x_0 - x_1\).
    \item 如果 \(r = 0\), 则返回 \(x = [x_0;x_1]\). 否则令 \(x_1 := \floor{\frac{1}{r}}, r := x_1 - x_2\).
    \item ...
    \item 直到精度满足要求, 返回 \(x = [x_0;x_1, \cdots, x_i]\).
\end{itemize}

\begin{theorem}
    有理数的简单连分数一定可以在有限步中计算出.
\end{theorem}