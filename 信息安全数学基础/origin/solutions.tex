\begin{problem}
    计算2,5,10模13的指数.
    \begin{answer}
        经计算:
        \[2^{12} \equiv 3^{3} \equiv 10^6 \equiv 1 \pmod{13}\]
        \[\ord{13}{2}=12,\quad \ord{13}{3}=3, \quad \ord{13}{10}=6\]
    \end{answer}
\end{problem}

\begin{problem}
    求模47的原根数量.
    \begin{answer}
        已知47为奇素数, 所以模47的原根存在, 原根数量为:
        \[\varphi(47-1)= \varphi(46) = \varphi(2)\cdot\varphi(23)= 1 \cdot 22 = 22\]
    \end{answer}
\end{problem}

\begin{problem}
    设 \(m = a^n - 1\), 其中 \(a\) 与 \(n\) 是正整数. 证明:
    \(\ord{m}{a} = n\), 从而得到 \(n \mid \varphi(m)\).
\begin{answer}
    容易得到 \(a \ne 1\), 且
    \[a^n = m + 1 \equiv 1 \pmod{m}\]
    由题设知\(a > 1\), 则\(\forall k(1 \le k < n): 1 < a^k < m+1\).
    因此\(n\)是满足\(a^n \equiv1 \pmod{m}\)的最小整数, 即
    \(\ord{m}{a} = n\).

    又由Euler公式得到\(a^{\varphi(m)} \equiv 1 \pmod{m}\).
    所以可以得到\(\ord{m}{a} \varphi(m)\)即\(n \mid \varphi(m)\)
\end{answer}
\end{problem}

\begin{problem}
    求模 \(167^2\) 的原根.
\begin{answer}
    已知167是素数, 所以\(167^2\)存在原根.先找167的原根\(g\).
    \(167 - 1 = 166 = 2 \cdot 83\).
    
    已知\(g\)满足\(g^{\frac{p-1}{q_i}} \not\equiv 1 \pmod{167}\),
    其中\(q_i = 2, 83\). 经过检验, \(g = 6\). 所以\(g_1 = g = 6\)或\(g_2 = g + p = 173\)是模167的原根.
    计算验证:
    \[g_1^{167-1} \equiv 1 + 114 \cdot 167, \quad g_2^{167-1} \equiv 1 + 86 \cdot 167 \pmod{167^2}\]
    因此\(g_1 = 6, g_2 = 173\)都是模\(167^2\)的原根.
\end{answer}
\end{problem}

\begin{problem}
    求解同余方程\(x^{22} \equiv 5 \pmod{41}\).
\begin{answer}
    已知模41的一个原根是6,且\(\gcd(22, \varphi(41)) = 2\), 
    可以得知这个方程解数为2.
    又\(\ind{6}{5} \equiv 22 \pmod{41} \).

    可以得出原方程等价于\(22\ind{6}{x} = \ind{6}{5}\pmod{40}\).解得\(\ind{6}{x} \equiv 1, 21 \pmod{40}\).
    
    因此\(x \equiv 6, 35 \pmod{41}\).
\end{answer}
\end{problem}

