\section{多项式环}

\begin{definition}[多项式环]
    整数环、有理数域、实数域上的全体多项式构成的\textbf{多项式环}:
    \begin{align*}
        \Z[x] = \{\sum_{i=0}^n a_i x^i | a_i \in \Z, n \ge 0\} \\
        \Q[x] = \{\sum_{i=0}^n a_i x^i | a_i \in \Q, n \ge 0\} \\
        \R[x] = \{\sum_{i=0}^n a_i x^i | a_i \in \R, n \ge 0\}
    \end{align*}
\end{definition}

\begin{definition}
    设 \(R\) 是一个整环. 系数取自 \(R\) 的全体多项式构成的集合:
    \begin{equation*}
        R[x]=\{\sum_{i=0}^n a_i x^i | a_i \in R, n \ge 0\}
    \end{equation*}
    则称 \(R[x]\) 是\textbf{多项式整环}.
\end{definition}

\begin{definition}
    设 \(f(x),g(x)\) 是多项式整环 \(R[x]\) 中的任意两个多项式, 其中 \(g(x) \ne 0\).
    如果存在多项式 \(q(x)\) 使得等式 
    \[f(x) = q(x) \cdot g(x)\]
    成立,就称 \(g(x)\) \textbf{整除} \(f(x)\), 记为 \(g(x) \mid f(x)\).
\end{definition}

\begin{definition}[不可约多项式]
    设 \(f(x)\) 是整环 \(R\) 上的非常数多项式. 如果除了平凡因式 \(f(x)\) 以外,
    \(f(x)\) 没有其他非常数多项式, 那么, \(f(x)\) 就称为 \textbf{不可约多项式};
    否则称为可约多项式.
\end{definition}

\textit{例子: \(4x^2+4\) 是一个不可约多项式.}

\begin{theorem}
    设 \(f(x)\) 是域 \(K\) 上的次数为 \(n\) 的可约多项式, \(p(x)\) 是 \(f(x)\) 的
    次数最小的非常数因式. 则 \(p(x)\) 一定是不可约多项式, 且
    \[\deg p < \frac{1}{2}\deg f\]
\end{theorem}

\begin{theorem}
    设 \(f(x)\) 是域 \(K\) 上的多项式, 如果 
    \(\forall p(x)\) 满足 \(\deg p < \frac{1}{2}\deg f\)
    且 \(p(x)\) 不可约, 都有: \(p(x) \nmid f(x)\),
    则 \(f(x)\) 一定是不可约多项式.
\end{theorem}

\begin{example}
    \(f(x) = x^8 + x^4 + x^3 + x + 1\) 是 \(\F_2[x]\) 中的不可约多项式.
\end{example}

\begin{definition}[多项式的Euclid除法]
    给定整环上的多项式 \(f(x),g(x) (\deg f \ge \deg g)\),
    那么可以找到两个多项式 \(q(x), r(x)\) 使得
    \[f(x) = q(x) \cdot g(x) + r(x)\]
    且 \(\deg r < \deg g\).
\end{definition}

\begin{definition}[最大公因式, 最小公倍式]
    设 \(f(x),g(x),d(x)\) 是整环 \(R\) 上的多项式. 称 \(d(x)\) 是
    \(f(x),g(x)\) 的 \textbf{最大公因式}, 如果
    \[d(x) \mid f(x), \quad d(x) \mid g(x)\]
    并且 \(\forall h(x) : h(x) \mid f(x), h(x) \mid g(x)\) 都有 \(h(x) \mid d(x)\).

    称 \(m(x)\) 是 \(f(x),g(x)\) 的 \textbf{最小公倍式}, 如果
    \[f(x) \mid m(x), \quad g(x) \mid m(x)\]
    并且 \(\forall h(x) : f(x) \mid h(x), g(x) \mid h(x)\) 都有 \(m(x) \mid h(x)\).

    \(d(x),m(x)\) 都可以记为 \(\gcd(f(x),g(x)),\lcm(f(x),g(x))\).
\end{definition}

\begin{definition}[多项式互素]
    设 \(f(x),g(x),d(x)\) 是整环 \(R\) 上的多项式.
    若 \(\gcd(f(x), g(x)) = 1\),则称 \(f(x)\) 与 \(g(x)\) 互素.
    记为 \(f(x) \perp g(x)\)
\end{definition}

\begin{theorem}[多项式广义Euclid除法]
    设 \(f(x),g(x),d(x)\) 是域 \(K\) 上的多项式.
    \(\exists s_k(x),t_k(x)\) 使得
    \[s_k(x)f(x) + t_k(x)g(x) = \gcd(f(x),g(x))\]
    对于 \(i = 0,1,2,\cdots,k\). \(s_i(x),t_i(x)\)归纳定义为:
    \[\begin{cases}
        r_{-2}(x) = f(x),\quad r_{-1}(x) = g(x),\quad r_{i} = r_{i-2} \mod{r_{i-1}} \\
        q_{i} = \lfloor \frac{r_{i-2}}{r_{i-1}} \rfloor \\
        s_{-2}(x) = 1,\quad s_{-1}(x) = 0,\quad s_{i}(x) = -q_i(x) s_{i-1}+s_{i-2} \\
        t_{-2}(x) = 0,\quad s_{-1}(x) = 1,\quad t_{i}(x) = -q_i(x) t_{i-1}+t_{i-2} \\
    \end{cases}\]
\end{theorem}

还可以在域 \(K\) 上的多项式环 \(K[x]\) 上完美复刻第二章关于同余的知识点.

\begin{definition}[多项式环的商环]
    
\end{definition}

\begin{theorem}
    设 \(p(x)\) 是域 \(K\) 上的多项式环 \(K[x]\) 的一个不可约多项式, 
    则 \(K[x]\) 关于理想 \((p(x))\) 的商环 \(K[x]/(p(x))\) 关于
    多项式模 \(p(x)\) 加法以及模 \(p(x)\) 乘法构成一个域.
\end{theorem}
    